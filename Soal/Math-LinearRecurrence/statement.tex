\documentclass{article}

\usepackage{geometry}
\usepackage{amsmath}
\usepackage{graphicx, eso-pic}
\usepackage{listings}
\usepackage{hyperref}
\usepackage{multicol}
\usepackage{fancyhdr}
\pagestyle{fancy}
\fancyhf{}
\hypersetup{ colorlinks=true, linkcolor=black, filecolor=magenta, urlcolor=cyan}
\geometry{ a4paper, total={170mm,257mm}, top=10mm, right=20mm, bottom=20mm, left=20mm}
\setlength{\parindent}{0pt}
\setlength{\parskip}{0.3em}
\renewcommand{\headrulewidth}{0pt}

\rfoot{\thepage}
\fancyhf{} % sets both header and footer to nothing
\renewcommand{\headrulewidth}{0pt}
\lfoot{\textbf{CnC Intern Contest}}
\pagenumbering{gobble}

\fancyfoot[CE,CO]{\thepage}
\lstset{
    basicstyle=\ttfamily\small,
    columns=fixed,
    extendedchars=true,
    breaklines=true,
    tabsize=2,
    prebreak=\raisebox{0ex}[0ex][0ex]{\ensuremath{\hookleftarrow}},
    frame=none,
    showtabs=false,
    showspaces=false,
    showstringspaces=false,
    prebreak={},
    keywordstyle=\color[rgb]{0.627,0.126,0.941},
    commentstyle=\color[rgb]{0.133,0.545,0.133},
    stringstyle=\color[rgb]{01,0,0},
    captionpos=t,
    escapeinside={(\%}{\%)}
}

\begin{document}

\begin{center}

    
    \section*{Golden Experience Requiem} % ganti judul soal

    \begin{tabular}{ | c c | }
        \hline
        Batas Waktu  & 1s \\    % jangan lupa ganti time limit
        Batas Memori & 256MB \\  % jangan lupa ganti memory limit
        \hline
    \end{tabular}
\end{center}

\subsection*{Deskripsi}
Di kota Venice, tinggal seseorang yang bernama Giorno Giovanna. Ia adalah seorang remaja. Suatu hari, ia mendapat kabar bahwa semua temannya telah diculik oleh perusahaan Uncover Corp. Tanpa basa-basi, ia pun bergegas langsung menemui CEO dari perusahaan tersebut, yaitu pak Yadu. Pak Yadu memiliki stand yang bernama Holoro. Stand pak Yadu dapat membuat lubang - lubang portal di dunia nyata dan menghisap kamu sehingga kamu masuk ke trap-nya pak Yadu. Pak Yadu yang sudah tahu bahwa Giorno akan menemui dia langsung mengeluarkan stand-nya sambil berteriak\\

"WWWWWWWWWWWWWWWWWWRRRRRRRRRRRRRRRRRRRRRRYYYYYYYYYYYYYYYYYY\\
YYYYYYYYYYYYYYYYYYYYYYYYYYYYYYYYYYYYYYYYYYYYYYYYYYYYYYYYYYYYYYYYY\\
YYYYYYYYYYYYYYYYYYYYYYYYYYYYYYYYYYYYYYYYYYYYYYYYYYYYYYYYYYYYYYYYY\\
YYYYYYYYYYYYYYYYYYYYYYYYYYYYYYYYYYYYYYYYYYYYYYYYYYYY!!!!!!!"\\

Giorno langsung mengeluarkan standnya Golden Experience Requiem ( GER ). GER dapat mengalahkan pak Yadu dalam 1 serangan. GER sekarang memiliki level $M$. Artinya, dia dapat mengeluarkan kekuatan dengan mulai dari level $0$ sampai $( M - 1 )$. Giorno dapat mengalahkan pak Yadu dengan menggunakan keuatan pada level $N$. Apabila $N > M-1$, maka kekuatan pada level $N$ ( kita anggap $F_N$ sebagai kekuatan pada level N ) adalah

\[F_N = \frac{F_{N-1}}{K_1} + \frac{F_{N-2}}{K_2} + \frac{F_{N-3}}{K_3} + ... +\frac{F_{N-M}}{K_M} \]

\\
Carilah nilai dari kekuatan pada level N tersebut!. Karena sulit, maka nilai kekuatan itu harus di modulo $10^9 + 7$

\subsection*{Format Masukan}

Baris pertama terdiri dari satu bilangan bulat positif $N$ ($1 \leq N \leq 10.000.000.000.000.000.000$), yang menyatakan level dari kekuatan yang ingin kita cari.\\
Baris kedua berisi satu bilangan bulat positif $M$ ($1 \leq M \leq 100$), yang menyatakan banyak level kita saat ini.\\
Baris ketiga berisi $M$ bilangan bulat positif yang menyatakan $F_1$ sampai $F_{M}$ ($F_i \leq 1.000.000.000$).\\
Baris keempat berisi $M$ bilangan bulat positif yang menyatakan $K_1$ sampai $K_{M}$ ($ 1 < K_i \leq 1.000.000.000 $).\\

\subsection*{Format Keluaran}

Keluarkan nilai dari kekuatan pada level ke N di modulo $10^9 + 7$
\\

\begin{multicols}{2}
\subsection*{Contoh Masukan}
\begin{lstlisting}
4
3
8 4 4
2 2 1
\end{lstlisting}
\columnbreak
\subsection*{Contoh Keluaran}
\begin{lstlisting}
12
\end{lstlisting}
\vfill
\null
\end{multicols}

\subsection*{Penjelasan}
\(F_4   = \frac{4}{2} + \frac{4}{2} + \frac{8}{1}\) = 10\\


\pagebreak

\end{document}