\documentclass{article}

\documentclass{article}

\usepackage{geometry}
\usepackage{amsmath}
\usepackage{graphicx, eso-pic}
\usepackage{listings}
\usepackage{hyperref}
\usepackage{multicol}
\usepackage{fancyhdr}
\pagestyle{fancy}
\fancyhf{}
\hypersetup{ colorlinks=true, linkcolor=black, filecolor=magenta, urlcolor=cyan}
\geometry{ a4paper, total={170mm,257mm}, top=10mm, right=20mm, bottom=20mm, left=20mm}
\setlength{\parindent}{0pt}
\setlength{\parskip}{0.3em}
\renewcommand{\headrulewidth}{0pt}

\rfoot{\thepage}
\fancyhf{} % sets both header and footer to nothing
\renewcommand{\headrulewidth}{0pt}
\lfoot{\textbf{CnC Intern Contest}}
\pagenumbering{gobble}

\fancyfoot[CE,CO]{\thepage}
\lstset{
    basicstyle=\ttfamily\small,
    columns=fixed,
    extendedchars=true,
    breaklines=true,
    tabsize=2,
    prebreak=\raisebox{0ex}[0ex][0ex]{\ensuremath{\hookleftarrow}},
    frame=none,
    showtabs=false,
    showspaces=false,
    showstringspaces=false,
    prebreak={},
    keywordstyle=\color[rgb]{0.627,0.126,0.941},
    commentstyle=\color[rgb]{0.133,0.545,0.133},
    stringstyle=\color[rgb]{01,0,0},
    captionpos=t,
    escapeinside={(\%}{\%)}
}

\begin{document}

\begin{center}

    
    \section*{Baskara Seorang Matematikawan} % ganti judul soal

    \begin{tabular}{ | c c | }
        \hline
        Batas Waktu  & 1s \\    % jangan lupa ganti time limit
        Batas Memori & 256MB \\  % jangan lupa ganti memory limit
        \hline
    \end{tabular}
\end{center}

\subsection*{Deskripsi}

Baskara adalah seorang matematikawan yang handal. Oleh karena itu, dia diminta untuk mencari nilai x yang memenuhi persamaan polinom

\[ f(x) = ax^3 + bx^2 + cx + d \]

Jumlah persamaan yang diberikan sangat banyak. Baskara adalah orang yang sangat sibuk sekali. Oleh karena itu, dia meminta bantuan anda untuk membuat program untuk mencari nilai x tersebut. Buatlah program untuk membantu Baskara!

\subsection*{Format Masukan}

Baris pertama terdiri dari satu bilangan bulat positif $T$ ($1 \leq T \leq 100.000$), menyatakan banyaknya kasus uji. \\
$T$ baris berikutnya terdiri dari 4 bilangan bulat, dengan baris ke-$i$ menyatakan bilangan $A_i$, $B_i$, $C_i$, $D_i$  $(0 \leq A_i,B_i,C_i,D_i \leq 100 $ dan $ A_i + B_i + C_i \geq 1)$  dan $f(x)$ $(0 \leq f(x) \leq 10^{15} )$\\
Dipastikan akan ada bilangan x yang memenuhi $f(x)$ pada setiap kasus uji

\subsection*{Format Keluaran}

Untuk tiap kasus uji, tuliskan $T$ baris, dengan baris ke-$i$ menyatakan nilai x yang memenuhi $f(x)$ tersebut.
Output bilangan bulat positif x yang memenuhi $f(x)$, dapat dipastikan hanya ada satu jawaban x yang memenuhi dan x bisa bernilai 0.  
\\

\begin{multicols}{2}
\subsection*{Contoh Masukan 1}
\begin{lstlisting}
1
1 2 3 5 27
\end{lstlisting}

\subsection*{Contoh Masukan 2}
\begin{lstlisting}
3
0 0 1 5 7
1 0 0 3 30
12 23 34 0 0
\end{lstlisting}

\subsection*{Contoh Masukan 2}
\begin{lstlisting}
1
0 1 0 1 2
\end{lstlisting}

\columnbreak
\subsection*{Contoh Keluaran 1}
\begin{lstlisting}
2
\end{lstlisting}

\subsection*{Contoh Keluaran 2}
\begin{lstlisting}
2
3
0
\end{lstlisting}

\subsection*{Contoh Keluaran 3}
\begin{lstlisting}
1
\end{lstlisting}
\end{multicols}


\subsection*{Penjelasan}
Pada test case pertama, fungsinya menjadi \(f(x) = 1x^3 + 2x^2 + 3x + 5\). Kita juga mengetahui nilai dari \(f(x)\). Oleh karena itu, kita dapat menemukan persamaan
\[27 = 1x^3 + 2x^2 + 3x + 5\]

Nilai x yang memenuhi adalah 2

Pada test case ketiga, ada 2 nilai x yang memenuhi yaitu 1 dan -1, tetapi karena jawaban yang diterima hanya bilangan bulat postif maka jawabannya adalah 1


\pagebreak

\end{document}
