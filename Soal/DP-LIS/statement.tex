\documentclass{article}

\usepackage{geometry}
\usepackage{amsmath}
\usepackage{graphicx, eso-pic}
\usepackage{listings}
\usepackage{hyperref}
\usepackage{multicol}
\usepackage{fancyhdr}
\pagestyle{fancy}
\fancyhf{}
\hypersetup{ colorlinks=true, linkcolor=black, filecolor=magenta, urlcolor=cyan}
\geometry{ a4paper, total={170mm,257mm}, top=10mm, right=20mm, bottom=20mm, left=20mm}
\setlength{\parindent}{0pt}
\setlength{\parskip}{0.3em}
\renewcommand{\headrulewidth}{0pt}

\rfoot{\thepage}
\fancyhf{} % sets both header and footer to nothing
\renewcommand{\headrulewidth}{0pt}
\lfoot{\textbf{CnC Intern Contest}}
\pagenumbering{gobble}

\fancyfoot[CE,CO]{\thepage}
\lstset{
    basicstyle=\ttfamily\small,
    columns=fixed,
    extendedchars=true,
    breaklines=true,
    tabsize=2,
    prebreak=\raisebox{0ex}[0ex][0ex]{\ensuremath{\hookleftarrow}},
    frame=none,
    showtabs=false,
    showspaces=false,
    showstringspaces=false,
    prebreak={},
    keywordstyle=\color[rgb]{0.627,0.126,0.941},
    commentstyle=\color[rgb]{0.133,0.545,0.133},
    stringstyle=\color[rgb]{01,0,0},
    captionpos=t,
    escapeinside={(\%}{\%)}
}

\begin{document}

\begin{center}

    
    \section*{Si Kontraktor} % ganti judul soal

    \begin{tabular}{ | c c | }
        \hline
        Batas Waktu  & 1s \\    % jangan lupa ganti time limit
        Batas Memori & 256MB \\  % jangan lupa ganti memory limit
        \hline
    \end{tabular}
\end{center}

\subsection*{Deskripsi}
Terdapat seorang kontraktor yang ingin membuat tower dari balok batu. Ia ingin membuat tower tersebut setinggi mungkin. Dia tahu bahwa apabila ia menumpuk batu A diatas batu B, maka panjang dan lebar batu B harus lebih lebar dari batu A. Bantu kontraktor tersebut untuk mencari tinggi maksimum dari tower tersebut!


\subsection*{Format Masukan}

Baris pertama terdiri dari satu bilangan bulat positif $N$ ($1 \leq N \leq 10.000$), menyatakan banyaknya balok\\
$N$ baris berikutnya terdiri dari 3 bilangan, dengan baris ke-$i$ menyatakan bilangan $P_i$, $L_i$, dan $T_i$ ($1 \leq P_i, L_i, T_i \leq 100.000$) yang menyatakan panjang, lebar, dan tinggi dari balok ke $i$ \\

\subsection*{Format Keluaran}

Keluarkan tinggi maksimum
\\

\begin{multicols}{2}
\subsection*{Contoh Masukan}
\begin{lstlisting}
6
1 5 4
1 2 2
2 3 2
2 4 1
3 6 2
4 5 3
\end{lstlisting}
\columnbreak
\subsection*{Contoh Keluaran}
\begin{lstlisting}
7
\end{lstlisting}
\vfill
\null
\end{multicols}

\subsection*{Penjelasan}
Kita dapat mengonnstruksi tower tertinggi menggunakan balok ke 2, 3, dan 7.\\
Terurut dari bawah ke atas, berikutlah susunan tower tersebut.\\

4 5 3\\
2 3 2\\
1 2 2\\

tinggi tower tersebut adalah 7.

\pagebreak

\end{document}