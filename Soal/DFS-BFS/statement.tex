\documentclass{article}

\usepackage{geometry}
\usepackage{amsmath}
\usepackage{graphicx, eso-pic}
\usepackage{listings}
\usepackage{hyperref}
\usepackage{multicol}
\usepackage{fancyhdr}
\pagestyle{fancy}
\fancyhf{}
\hypersetup{ colorlinks=true, linkcolor=black, filecolor=magenta, urlcolor=cyan}
\geometry{ a4paper, total={170mm,257mm}, top=10mm, right=20mm, bottom=20mm, left=20mm}
\setlength{\parindent}{0pt}
\setlength{\parskip}{0.3em}
\renewcommand{\headrulewidth}{0pt}

\rfoot{\thepage}
\fancyhf{} % sets both header and footer to nothing
\renewcommand{\headrulewidth}{0pt}
\lfoot{\textbf{CnC Intern Contest}}
\pagenumbering{gobble}

\fancyfoot[CE,CO]{\thepage}
\lstset{
    basicstyle=\ttfamily\small,
    columns=fixed,
    extendedchars=true,
    breaklines=true,
    tabsize=2,
    prebreak=\raisebox{0ex}[0ex][0ex]{\ensuremath{\hookleftarrow}},
    frame=none,
    showtabs=false,
    showspaces=false,
    showstringspaces=false,
    prebreak={},
    keywordstyle=\color[rgb]{0.627,0.126,0.941},
    commentstyle=\color[rgb]{0.133,0.545,0.133},
    stringstyle=\color[rgb]{01,0,0},
    captionpos=t,
    escapeinside={(\%}{\%)}
}

\begin{document}

\begin{center}

    
    \section*{Luas Jaring} % ganti judul soal

    \begin{tabular}{ | c c | }
        \hline
        Batas Waktu  & 1 s \\    % jangan lupa ganti time limit
        Batas Memori & 256 MB \\  % jangan lupa ganti memory limit
        \hline
    \end{tabular}
\end{center}

\subsection*{Deskripsi}

Semisal kita dapat merepresentasikan jaring dengan array 0 dan 1. Panjang jaring adalah panjang elemen 1 yang terhubung (termasuk diagonal).

\\
\\
Contoh representasi jaring:
\begin{center}
    
1 & 1 & 1 \\
1 & 0 & 0 \\
0 & 1 & 0 \\
0 & 0 & 0

\end{center}

Panjang jaring terpanjang adalah 5.

\begin{center}
    
1 & 0 & 1 \\
1 & 0 & 0 \\
0 & 1 & 0 \\
0 & 0 & 0

\end{center}

Panjang jaring terpanjang adalah 3.

\subsection*{Format Masukan}

Baris pertama terdiri dari dua bilangan bulat positif $N$ ($1 \leq N)$ dan $M$ ($1 \leq M \leq 500$), menyatakan ukuran array N*M.
Selanjutnya input meminta N baris bilangan 0 atau 1, yang tiap barisnya terdapat M bilangan, merepresentasikan isi array.

\subsection*{Format Keluaran}

Tuliskan panjang jaring terpanjang pada jaring tersebut.

\begin{multicols}{2}
\subsection*{Contoh Masukan}
\begin{lstlisting}
4 3
1 1 1
1 0 0
0 1 0
0 0 0
\end{lstlisting}
\columnbreak
\subsection*{Contoh Keluaran}
5
\vfill
\null
\end{multicols}
\pagebreak

\end{document}