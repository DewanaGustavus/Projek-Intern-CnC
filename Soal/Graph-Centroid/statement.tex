\documentclass{article}

\usepackage{geometry}
\usepackage{amsmath}
\usepackage{graphicx, eso-pic}
\usepackage{listings}
\usepackage{hyperref}
\usepackage{multicol}
\usepackage{fancyhdr}
\pagestyle{fancy}
\fancyhf{}
\hypersetup{ colorlinks=true, linkcolor=black, filecolor=magenta, urlcolor=cyan}
\geometry{ a4paper, total={170mm,257mm}, top=10mm, right=20mm, bottom=20mm, left=20mm}
\setlength{\parindent}{0pt}
\setlength{\parskip}{0.3em}
\renewcommand{\headrulewidth}{0pt}

\rfoot{\thepage}
\fancyhf{} % sets both header and footer to nothing
\renewcommand{\headrulewidth}{0pt}
\lfoot{\textbf{CnC Intern Contest}}
\pagenumbering{gobble}

\fancyfoot[CE,CO]{\thepage}
\lstset{
    basicstyle=\ttfamily\small,
    columns=fixed,
    extendedchars=true,
    breaklines=true,
    tabsize=2,
    prebreak=\raisebox{0ex}[0ex][0ex]{\ensuremath{\hookleftarrow}},
    frame=none,
    showtabs=false,
    showspaces=false,
    showstringspaces=false,
    prebreak={},
    keywordstyle=\color[rgb]{0.627,0.126,0.941},
    commentstyle=\color[rgb]{0.133,0.545,0.133},
    stringstyle=\color[rgb]{01,0,0},
    captionpos=t,
    escapeinside={(\%}{\%)}
}

\begin{document}

\begin{center}

    
    \section*{Xenia dan Tree} % ganti judul soal

    \begin{tabular}{ | c c | }
        \hline
        Batas Waktu  & 5s \\    % jangan lupa ganti time limit
        Batas Memori & 256MB \\  % jangan lupa ganti memory limit
        \hline
    \end{tabular}
\end{center}

\subsection*{Deskripsi}
Xenia adalah seorang programmer yang mempunyai tree yang terdiri dari $n$ node. Node-node dari tree tersebut indexnya adalah 1 sampai $n$, dan node pertama diwarnai dengan warna merah dan sisanya diwarnai dengan warna biru. Jarak antara dua node $v$ dan $u$ adalah banyak edge pada path yang terpendek antara $u$ dan $v$.
Xenia harus mencari cara tercepat untuk mengeksekusi dua tipe query berikut:
\begin{enumerate}
    \setlength\itemsep{0pt}
    \item Mewarnai node berwarna biru tertentu dengan warna merah.
    \item Menentukan node merah dengan distance terdekat dari node merah tertentu dan menuliskan ke layar jarak terpendek dari node berwarna merah terdekat.
\end{enumerate}
\subsection*{Format Masukan}
Baris pertama  input berisi 2 integer $n$ dan $m$ ($2 \leq n \leq 10^5$, $1 \leq m \leq 10^5$), masing-masing adalah jumlah node dari tree dan jumlah query dari tree. Baris n-1 berikutnya berisi edge dari tree tersebut, baris ke-i berisi 2 integer yaitu $a_i$ dan $b_i$ ($1 \leq a_i, b_i \leq n, a_i \neq b_i$) yang merupakan edge dari tree tersebut.
\\
\\
m baris berikutnya berisi query. Setiap query berisi sepasang integer yaitu $t_i$ dan $v_i$ ($1 \leq t_i \leq 2$, $1 \leq v_i \leq n$). Jika $t_i$ = 1, maka program harus mewarnai node $v_i$ yang berwarna biru dengan warna merah. Jika $t_i$ = 2, maka program harus menuliskan jarak terdekat dari node warna merah ke node $v_i$.
\subsection*{Format Keluaran}
Untuk setiap query yang tipe ke-2, tuliskan outputnya dalam satu baris
\\

\begin{multicols}{2}
\subsection*{Contoh Masukan}
\begin{lstlisting}
5 4
1 2
2 3
2 4
4 5
2 1
2 5
1 2
2 5
\end{lstlisting}
\columnbreak
\subsection*{Contoh Keluaran}
\begin{lstlisting}
0
3
2
\end{lstlisting}
\vfill
\null
\end{multicols}

\subsection*{Penjelasan}
 

\pagebreak

\end{document}