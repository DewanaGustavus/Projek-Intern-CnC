\documentclass{article}

\usepackage{geometry}
\usepackage{amsmath}
\usepackage{graphicx, eso-pic}
\usepackage{listings}
\usepackage{hyperref}
\usepackage{multicol}
\usepackage{fancyhdr}
\pagestyle{fancy}
\fancyhf{}
\hypersetup{ colorlinks=true, linkcolor=black, filecolor=magenta, urlcolor=cyan}
\geometry{ a4paper, total={170mm,257mm}, top=10mm, right=20mm, bottom=20mm, left=20mm}
\setlength{\parindent}{0pt}
\setlength{\parskip}{0.3em}
\renewcommand{\headrulewidth}{0pt}

\rfoot{\thepage}
\fancyhf{} % sets both header and footer to nothing
\renewcommand{\headrulewidth}{0pt}
\lfoot{\textbf{CnC Intern Contest}}
\pagenumbering{gobble}

\fancyfoot[CE,CO]{\thepage}
\lstset{
    basicstyle=\ttfamily\small,
    columns=fixed,
    extendedchars=true,
    breaklines=true,
    tabsize=2,
    prebreak=\raisebox{0ex}[0ex][0ex]{\ensuremath{\hookleftarrow}},
    frame=none,
    showtabs=false,
    showspaces=false,
    showstringspaces=false,
    prebreak={},
    keywordstyle=\color[rgb]{0.627,0.126,0.941},
    commentstyle=\color[rgb]{0.133,0.545,0.133},
    stringstyle=\color[rgb]{01,0,0},
    captionpos=t,
    escapeinside={(\%}{\%)}
}

\begin{document}

\begin{center}

    
    \section*{Array dan Operasi} % ganti judul soal

    \begin{tabular}{ | c c | }
        \hline
        Batas Waktu  & 1s \\    % jangan lupa ganti time limit
        Batas Memori & 256MB \\  % jangan lupa ganti memory limit
        \hline
    \end{tabular}
\end{center}

\subsection*{Deskripsi}
Terdapat sebuah array $A$ yang berukuran $N$ dan $M$ buah pasangan baik dimana pasangan ($i,j$) dikatakan baik apabila $i_k + j_k$ adalah bilangan ganjil dan $1 \leq k \leq M$ dan $1 \leq i_k < j_k \leq N$. Dalam satu operasi, anda akan melakukan ini :
\begin{enumerate}
    \item Ambil 1 pasangan baik ($i_k, j_k$) dan 1 bilangan bulat positif lainnya ($V > 1$), dimana $V$ habis membagi $A[i_k]$ dan $A[j_k]$
    \item membagi $A[i_k]$ dan $A[j_k]$ dengan $V$
\end{enumerate}
Cari jumlah operasi maksimum yang bisa anda lakukan terhadap array tersebut
\subsection*{Format Masukan}

Baris pertama terdiri dari dua bilangan positif $N$ ($2 \leq N \leq 100$) dan $M$ ($1 \leq M \leq 100$)\\
Baris kedua terdiri atas $N$ buah bilangan yang menyatakan $A[1], A[2], A[3], ... ,A[n]$ ($1 \leq A[i] \leq N$)\\
Masing - masing baris pada setiap $M$ baris selanjutnya terdiri atas pasangan baik. Setiap baris terdiri atas $i_k$ dan $j_k$ ($1 \leq k \leq M,1 \leq i_k < j_k \leq N,$ $i_k$ + $j_k$ dipastikam  merupakan bilangan ganjil) 

\subsection*{Format Keluaran}

jumlah operasi maksimum yang bisa dilakukan.
\\

\begin{multicols}{2}
\subsection*{Contoh Masukan}
\begin{lstlisting}
3 2
8 12 8
1 2
2 3

\end{lstlisting}
\columnbreak
\subsection*{Contoh Keluaran}
\begin{lstlisting}
2
\end{lstlisting}
\vfill
\null
\end{multicols}

\subsection*{Penjelasan}
Berikut adalah operasi yang bisa kita lakukan
\begin{enumerate}
    \item Pada operasi pertama, kita dapat memilih pasangan 1 dan 2 serta memilih $V$ yang bernilai 2. Nilai dari $A[1]$ menjadi 4 dan nilai dari $A[2]$ menjadi 6.
    \item Pada operasi kedua, kita dapat memilih pasangan 1 dan 2 lagi serta memilih $V$ yang bernilai 2. Nilai dari $A[1]$ menjadi 2 dan nilai dari $A[2]$ menjadi 3.
\end{enumerate}
Kita tidak bisa melakukan operasi lainnya baik pada pasangan baik pertama maupun pasangan baik kedua

\pagebreak

\end{document}