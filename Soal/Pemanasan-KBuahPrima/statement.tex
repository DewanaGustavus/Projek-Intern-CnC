\documentclass{article}

\usepackage{geometry}
\usepackage{amsmath}
\usepackage{graphicx, eso-pic}
\usepackage{listings}
\usepackage{hyperref}
\usepackage{multicol}
\usepackage{fancyhdr}
\pagestyle{fancy}
\fancyhf{}
\hypersetup{ colorlinks=true, linkcolor=black, filecolor=magenta, urlcolor=cyan}
\geometry{ a4paper, total={170mm,257mm}, top=10mm, right=20mm, bottom=20mm, left=20mm}
\setlength{\parindent}{0pt}
\setlength{\parskip}{0.3em}
\renewcommand{\headrulewidth}{0pt}

\rfoot{\thepage}
\fancyhf{} % sets both header and footer to nothing
\renewcommand{\headrulewidth}{0pt}
\lfoot{\textbf{CnC Intern Contest}}
\pagenumbering{gobble}

\fancyfoot[CE,CO]{\thepage}
\lstset{
    basicstyle=\ttfamily\small,
    columns=fixed,
    extendedchars=true,
    breaklines=true,
    tabsize=2,
    prebreak=\raisebox{0ex}[0ex][0ex]{\ensuremath{\hookleftarrow}},
    frame=none,
    showtabs=false,
    showspaces=false,
    showstringspaces=false,
    prebreak={},
    keywordstyle=\color[rgb]{0.627,0.126,0.941},
    commentstyle=\color[rgb]{0.133,0.545,0.133},
    stringstyle=\color[rgb]{01,0,0},
    captionpos=t,
    escapeinside={(\%}{\%)}
}

\begin{document}

\begin{center}

    
    \section*{K Buah Prima} % ganti judul soal

    \begin{tabular}{ | c c | }
        \hline
        Batas Waktu  & 1s \\    % jangan lupa ganti time limit
        Batas Memori & 256MB \\  % jangan lupa ganti memory limit
        \hline
    \end{tabular}
\end{center}

\subsection*{Deskripsi}
Dengklek mempunyai suatu bilangan K, L, dan R. Dia tahu bahwa setiap bilangan bulat positif dapat di bentuk dari perkalian dari satu atau lebih bilangan prima (\href{https://mathworld.wolfram.com/FundamentalTheoremofArithmetic.html}{Teori Fundamental Aritmatika} ).Dia ingin mencari tahu berapa banyak bilangan yang berada di interval [L,R] dan memiliki tepat K buah pembagi prima yang berbeda. Bantu dengklek mencari berapa total bilangan tersebut!

\subsection*{Format Masukan}

Baris pertama terdiri atas 3 buah bilangan yaitu $K$ ($1 \leq K \leq 10^9$), $L$ ($1 \leq L \leq R$), dan $R$ ($L \leq R \leq 10.000$)

\subsection*{Format Keluaran}

Keluarkan jumlah bilangan yang mempunyai k-buah pembagi prima yang berbeda.
\\

\begin{multicols}{2}
\subsection*{Contoh Masukan}
\begin{lstlisting}
2 6 15
\end{lstlisting}
\columnbreak
\subsection*{Contoh Keluaran}
\begin{lstlisting}
5
\end{lstlisting}
\vfill
\null
\end{multicols}

\subsection*{Penjelasan}
Berikut adalah daftar bilangan bilangan tersebut :
\begin{itemize}
    \item $6  = 2   \times 3$
    \item $10 = 2   \times 5$
    \item $12 = 2^2 \times 3$
    \item $14 = 2   \times 7$
    \item $15 = 3   \times 5$
\end{itemize}

\pagebreak

\end{document}