\documentclass{article}

\usepackage{geometry}
\usepackage{amsmath}
\usepackage{graphicx, eso-pic}
\usepackage{listings}
\usepackage{hyperref}
\usepackage{multicol}
\usepackage{fancyhdr}
\pagestyle{fancy}
\fancyhf{}
\hypersetup{ colorlinks=true, linkcolor=black, filecolor=magenta, urlcolor=cyan}
\geometry{ a4paper, total={170mm,257mm}, top=10mm, right=20mm, bottom=20mm, left=20mm}
\setlength{\parindent}{0pt}
\setlength{\parskip}{0.3em}
\renewcommand{\headrulewidth}{0pt}

\rfoot{\thepage}
\fancyhf{} % sets both header and footer to nothing
\renewcommand{\headrulewidth}{0pt}
\lfoot{\textbf{CnC Intern Contest}}
\pagenumbering{gobble}

\fancyfoot[CE,CO]{\thepage}
\lstset{
    basicstyle=\ttfamily\small,
    columns=fixed,
    extendedchars=true,
    breaklines=true,
    tabsize=2,
    prebreak=\raisebox{0ex}[0ex][0ex]{\ensuremath{\hookleftarrow}},
    frame=none,
    showtabs=false,
    showspaces=false,
    showstringspaces=false,
    prebreak={},
    keywordstyle=\color[rgb]{0.627,0.126,0.941},
    commentstyle=\color[rgb]{0.133,0.545,0.133},
    stringstyle=\color[rgb]{01,0,0},
    captionpos=t,
    escapeinside={(\%}{\%)}
}

\begin{document}

\begin{center}

    
    \section*{Genap Tetapi Prima} % ganti judul soal

    \begin{tabular}{ | c c | }
        \hline
        Batas Waktu  & 2s \\    % jangan lupa ganti time limit
        Batas Memori & 256MB \\  % jangan lupa ganti memory limit
        \hline
    \end{tabular}
\end{center}

\subsection*{Deskripsi}
Sebuah bilangan dikatakan prigenap apabila bilangan tersebut merupakan perkalian antara suatu bilangan prima dan bilangan genap. Tentukanlah bilangan prigenap yang ke-n. Pada soal ini, bilangan 2 dianggap bukan prima :D.

\subsection*{Format Masukan}

Baris pertama terdiri dari satu bilangan bulat positif $T$ ($1 \leq T \leq 100.000$), menyatakan banyaknya kasus uji.\\
$T$ baris selanjutnya diisi oleh bilangan integer $N$ ($1 \leq N \leq 100.000$) yang menyatakan indeks dari bilangan prigenap yang ingin kita bagi

\subsection*{Format Keluaran}

Untuk tiap kasus uji, tuliskan bilangan prigenapnya!
\\

\begin{multicols}{2}
\subsection*{Contoh Masukan}
\begin{lstlisting}
3
8
3
4
\end{lstlisting}
\columnbreak
\subsection*{Contoh Keluaran}
\begin{lstlisting}
26
12
14
\end{lstlisting}
\vfill
\null
\end{multicols}

\subsection*{Penjelasan}
Apabila bilangan primanya 3, maka prigenapnya : \\
$3 \times 2 = 6, 3\times  s4 = 12, 3\times6 = 18, 3\times 8 = 24 $

Apabila bilangan primanya 5, maka prigenapnya :\\
$5 \times 2 = 10, 5 \times 4 = 20, 5\times 6 = 30$

Apabila bilangan primanya 7, maka prigenapnya :\\
$7 \times 2 = 14, 7 \times 4 = 28, 7 \times 6 = 42$

Apabila bilangan primanya 11, maka prigenapnya : \\
$ 11 \times 2 = 22, 11 \times 4 = 44 $

Apabila bilangan primanya 13, maka prigenapnya :\\
$ 13 \times 2 = 26, 13 \times  4 = 52 $

Apabila bilangan primanya 17, maka prigenapnya : \\
$ 17 \times 2 = 34$

Apabila bilangan primannya 19, maka prigenapnya :\\
$ 19 \times 2 = 38. $

Maka kita dapat bilangan prigenap \\

$
1. 6 \\
2. 10 \\
3. 12\\
4. 14\\
5. 18\\
6. 20\\
7. 22\\
8. 26\\
$


\pagebreak

\end{document}