\documentclass{article}

\usepackage{geometry}
\usepackage{amsmath}
\usepackage{graphicx, eso-pic}
\usepackage{listings}
\usepackage{hyperref}
\usepackage{multicol}
\usepackage{fancyhdr}
\pagestyle{fancy}
\fancyhf{}
\hypersetup{ colorlinks=true, linkcolor=black, filecolor=magenta, urlcolor=cyan}
\geometry{ a4paper, total={170mm,257mm}, top=10mm, right=20mm, bottom=20mm, left=20mm}
\setlength{\parindent}{0pt}
\setlength{\parskip}{0.3em}
\renewcommand{\headrulewidth}{0pt}

\rfoot{\thepage}
\fancyhf{} % sets both header and footer to nothing
\renewcommand{\headrulewidth}{0pt}
\lfoot{\textbf{CnC Intern Contest}}
\pagenumbering{gobble}

\fancyfoot[CE,CO]{\thepage}
\lstset{
    basicstyle=\ttfamily\small,
    columns=fixed,
    extendedchars=true,
    breaklines=true,
    tabsize=2,
    prebreak=\raisebox{0ex}[0ex][0ex]{\ensuremath{\hookleftarrow}},
    frame=none,
    showtabs=false,
    showspaces=false,
    showstringspaces=false,
    prebreak={},
    keywordstyle=\color[rgb]{0.627,0.126,0.941},
    commentstyle=\color[rgb]{0.133,0.545,0.133},
    stringstyle=\color[rgb]{01,0,0},
    captionpos=t,
    escapeinside={(\%}{\%)}
}

\begin{document}

\begin{center}

    
    \section*{Poker} % ganti judul soal

    \begin{tabular}{ | c c | }
        \hline
        Batas Waktu  & 2s \\    % jangan lupa ganti time limit
        Batas Memori & 1024MB \\  % jangan lupa ganti memory limit
        \hline
    \end{tabular}
\end{center}

\subsection*{Deskripsi}
Terdapat kartu sebanyak 9$K$, untuk setiap $i$ = 1,2,..,9 terdapat K buah kartu yang tertulis $i$. Kita kemudian mengacak kartu tersebut dan memberikan 5 kartu kepada masing-masing Takahashi dan Aoki, 1 kartu dalam kondisi terbalik dan 4 sisanya tidak. Carilah kemungkinan dimana Takahashi menang, Takahashi menang jika kartu yang dia pegang bernilai lebih besar daripada kartu yang dipegang Aoki.
\\
\\
Kita dapat mendefinisikan nilai dari 5 kartu yang dipegang dengan:
$\displaystyle\sum_{i=1}^9i10^{c_i}$, dimana $c_i$ adalah jumlah kartu yang memiliki tulisan i.
\subsection*{Format Masukan}
Input berupa integer K ($2 \leq K \leq 10^5$) dan sepasang string. K melambangkan jumlah kartu yang tertulis $i$. String S melambangkan kartu Takahashi dan string T melambangkan kartu Aoki. S dan T masing-masing adalah string yang terdiri dari 5 karakter. Karakter terakhir stringnya dilambangkan dengan '#' yang melambangkan kartu yang terbalik.

\subsection*{Format Keluaran}
Tuliskan kemungkinan dimana Takahashi menang.

Jawaban akan terhitung benar jika jawaban tepat atau relatif errornya paling besar adalah $10^{-5}$
\\

\begin{multicols}{2}
\subsection*{Contoh Masukan}
\begin{lstlisting}
2
1144#
2233#
\end{lstlisting}
\columnbreak
\subsection*{Contoh Keluaran}
\begin{lstlisting}
0.4444444444444444
\end{lstlisting}
\vfill
\null
\end{multicols}

\subsection*{Penjelasan}
Contoh kemungkinan dimana Takahashi menang

Kartu Takahashi : 11446

Kartu Aoki : 22335

Nilai Takahashi : $1\cdot 10^2 + 2 \cdot 10^0 + 3\cdot 10^0 + 4\cdot10^2 + 5 \cdot 10^0 + 6 \cdot 10^1 + 7 \cdot 10^0 + 8 \cdot 10^0 + 9 \cdot 10^0 = 594$ 

Nilai Aoki : $1\cdot 10^0 + 2 \cdot 10^2 + 3\cdot 10^2 + 4\cdot10^0 + 5 \cdot 10^1 + 6 \cdot 10^0 + 7 \cdot 10^0 + 8 \cdot 10^0 + 9 \cdot 10^0 = 585$

Agar memperjelas mengenai banyak kemungkinan yang dapat terjadi, pada contoh untuk setiap angka terdapat 2 buah kartu, misal kartu yang ada mempunyai nomor 1A, 1B, 2A, 2B, dst.
Maka kasus dimana kartu Takahashi adalah 1A 1B 4A 4B (5A) dan kasus 1A 1B 4A 4B (5B) dianggap sebagai kasus yang berbeda.

\pagebreak

\end{document}
