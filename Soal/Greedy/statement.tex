\documentclass{article}

\usepackage{geometry}
\usepackage{amsmath}
\usepackage{graphicx, eso-pic}
\usepackage{listings}
\usepackage{hyperref}
\usepackage{multicol}
\usepackage{fancyhdr}
\pagestyle{fancy}
\fancyhf{}
\hypersetup{ colorlinks=true, linkcolor=black, filecolor=magenta, urlcolor=cyan}
\geometry{ a4paper, total={170mm,257mm}, top=10mm, right=20mm, bottom=20mm, left=20mm}
\setlength{\parindent}{0pt}
\setlength{\parskip}{0.3em}
\renewcommand{\headrulewidth}{0pt}

\rfoot{\thepage}
\fancyhf{} % sets both header and footer to nothing
\renewcommand{\headrulewidth}{0pt}
\lfoot{\textbf{CnC Intern Contest}}
\pagenumbering{gobble}

\fancyfoot[CE,CO]{\thepage}
\lstset{
    basicstyle=\ttfamily\small,
    columns=fixed,
    extendedchars=true,
    breaklines=true,
    tabsize=2,
    prebreak=\raisebox{0ex}[0ex][0ex]{\ensuremath{\hookleftarrow}},
    frame=none,
    showtabs=false,
    showspaces=false,
    showstringspaces=false,
    prebreak={},
    keywordstyle=\color[rgb]{0.627,0.126,0.941},
    commentstyle=\color[rgb]{0.133,0.545,0.133},
    stringstyle=\color[rgb]{01,0,0},
    captionpos=t,
    escapeinside={(\%}{\%)}
}

\begin{document}

\begin{center}

    
    \section*{Guru TK yang Kewalahan} % ganti judul soal

    \begin{tabular}{ | c c | }
        \hline
        Batas Waktu  & 1s \\    % jangan lupa ganti time limit
        Batas Memori & 256MB \\  % jangan lupa ganti memory limit
        \hline
    \end{tabular}
\end{center}

\subsection*{Deskripsi}
Ada seorang guru TK yang sangat kewalahan karena dia disuruh untuk memerintahkan murid-muridnya agar mereka duduk berdampingan. Karena murid-muridnya masih anak-anak, maka sudah pasti mereka duduk tidak beraturan. Beberapa murid ada yang duduk sendirian, ada juga yang duduk bersebelahan dengan teman-temannya. Karena guru TK tidak mau membuang tenaganya untuk memindahkan murid-muridnya, maka guru TK tersebut memikirkan cara agar dia bisa \textbf{meminimalisir} langkah pemindahan murid-muridnya ke kursinya masing-masing. Akan tetapi, setiap murid hanya mau pindah ke kursi yang tepat berada di sampingnya, jadi jika guru TK tersebut ingin memindahkan murid tersebut ke kursi ke-2 di kanannya maka murid tersebut harus pindah sebanyak 2 kali.

\subsection*{Format Masukan}
Input kondisi kursinya berupa string sepanjang $N$ ($1 \leq N \leq 1.000.000$). Kursi yang kosong dilambangkan dengan '.' dan kursi yang diduduki dilambangkan dengan 'x'.

\subsection*{Format Keluaran}
Buatlah sebuah algoritma yang dapat menentukan jumlah langkah minimal pemindahan tempat duduk agar para murid dapat duduk bersampingan.
\\

\begin{multicols}{2}
\subsection*{Contoh Masukan}
\begin{lstlisting}
....x..xx...x..
........xxxxxxx.xxxxxx.......x..
\end{lstlisting}
\columnbreak
\subsection*{Contoh Keluaran}
\begin{lstlisting}

5
14
\end{lstlisting}
\vfill
\null
\end{multicols}

\subsection*{Penjelasan}
 

\pagebreak

\end{document}