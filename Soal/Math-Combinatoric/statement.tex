\documentclass{article}

\usepackage{geometry}
\usepackage{amsmath}
\usepackage{graphicx, eso-pic}
\usepackage{listings}
\usepackage{hyperref}
\usepackage{multicol}
\usepackage{fancyhdr}
\pagestyle{fancy}
\fancyhf{}
\hypersetup{ colorlinks=true, linkcolor=black, filecolor=magenta, urlcolor=cyan}
\geometry{ a4paper, total={170mm,257mm}, top=10mm, right=20mm, bottom=20mm, left=20mm}
\setlength{\parindent}{0pt}
\setlength{\parskip}{0.3em}
\renewcommand{\headrulewidth}{0pt}

\rfoot{\thepage}
\fancyhf{} % sets both header and footer to nothing
\renewcommand{\headrulewidth}{0pt}
\lfoot{\textbf{CnC Intern Contest}}
\pagenumbering{gobble}

\fancyfoot[CE,CO]{\thepage}
\lstset{
    basicstyle=\ttfamily\small,
    columns=fixed,
    extendedchars=true,
    breaklines=true,
    tabsize=2,
    prebreak=\raisebox{0ex}[0ex][0ex]{\ensuremath{\hookleftarrow}},
    frame=none,
    showtabs=false,
    showspaces=false,
    showstringspaces=false,
    prebreak={},
    keywordstyle=\color[rgb]{0.627,0.126,0.941},
    commentstyle=\color[rgb]{0.133,0.545,0.133},
    stringstyle=\color[rgb]{01,0,0},
    captionpos=t,
    escapeinside={(\%}{\%)}
}

\begin{document}

\begin{center}

    
    \section*{Baju Anggota Tim} % ganti judul soal

    \begin{tabular}{ | c c | }
        \hline
        Batas Waktu  & 1 s \\    % jangan lupa ganti time limit
        Batas Memori & 256 MB \\  % jangan lupa ganti memory limit
        \hline
    \end{tabular}
\end{center}

\subsection*{Deskripsi}

Ada \textit{N} buah baju yang akan dibagikan kepada \textit{T} buah tim. Masing-masing tim memiliki $K_i$ orang anggota. Banyak anggota setiap tim tidak selalu sama dan $K_1 + K_2 + ... + K_T = N$.

Buatlah sebuah program yang dapat menghitung banyak kemungkinan susunan pembagian baju yang bisa kita dapatkan apabila kita membaginya kepada t buah tim tersebut di mod 1e9 + 7

\subsection*{Format Masukan}

Baris pertama terdiri dari satu bilangan bulat positif $j$ ($1 \leq j \leq 10.000$), menyatakan banyaknya kasus uji.
Tiap kasus uji diawali dengan 2 bilangan $N$ dan $T$ ($1 \leq T \leq N \leq  1.000$).
baris berikutnya terdiri dari $T$ bilangan, dengan bilangan ke-$i$ menyatakan bilangan $K_i$.

\subsection*{Format Keluaran}

Untuk tiap kasus uji, tuliskan sebuah bilangan, dengan bilangan baris ke-$i$ menyatakan banyak susunan pembagian baju untuk kasus ke-$i$.
\\

\begin{multicols}{2}
\subsection*{Contoh Masukan}
\begin{lstlisting}
1
5 3
3 1 1
\end{lstlisting}
\columnbreak
\subsection*{Contoh Keluaran}
\begin{lstlisting}
20
\end{lstlisting}
\vfill
\null
\end{multicols}

\pagebreak

\end{document}